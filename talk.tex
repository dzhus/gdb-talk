\documentclass[onlymath]{beamer}
\usepackage{fontspec}
\usepackage{xunicode,xltxtra}
\usepackage{polyglossia}
\usepackage{xltxtra}
\usepackage{fancybox}
\usepackage{graphicx}
\usepackage{listings}
\lstset{language=C, basicstyle=\ttfamily, frame=single}
\lstdefinelanguage{gdb}
{morekeywords={run,break,info,threads}}
\newcommand\tw\textwidth
\newcommand\neword\emph
\newcommand\code\texttt

\usefonttheme{professionalfonts}
\usetheme[secheader]{Boadilla}
\usecolortheme{whale}

\setsansfont[Mapping=tex-text]{Myriad Pro}
\setmonofont[Mapping=tex-text]{DejaVu Sans Mono}
\setmainlanguage{russian}

\title{GDB, Emacs и Google Summer of Code }
\author{Дмитрий Джус}
\institute{МГТУ им. Н.Э.Баумана}
\date{2009}

\begin{document}

\begin{frame}
  \titlepage
\end{frame}

\begin{frame}
  \frametitle{План}
  \tableofcontents
\end{frame}

\section{GDB}
\begin{frame}
  \frametitle{GNU Debugger — свободный отладчик}

  \begin{itemize}
  \item Переносимый отладчик для различных языков, включая C, C++ и Fortran
  \item Разработка начата rms в 1986 году в рамках проекта GNU и ныне
    продолжается силами сообщества при поддержке крупных компаний
  \item Стандартный отладчик для многих современных Unix-систем
  \item Используется в качестве компонента для ряда интегрированных
    сред разработки
  \end{itemize}
\end{frame}

\begin{frame}
  \frametitle{Командный интерпретатор GDB}
  Тут пример запуска и работы с GDB (текст прямо на слайде)
\end{frame}

\begin{frame}
  \frametitle{Новые возможности в GDB 7.0}
  \begin{itemize}
  \item Безостановочная отладка многопоточных приложений

  \item Поддержка Python в качестве языка расширения

  \item Обратимая отладка
  \end{itemize}
\end{frame}

\subsection{Отладка в режиме нон-стоп}
\begin{frame}
  \frametitle{Зачем нужна безостановочная отладка}
  \begin{itemize}
  \item Многопоточные приложения — тренд
  \item Нужно обеспечить недеструктивный контроль: отлаживать только
    ту нить, в которой проблемы
  \end{itemize}
\end{frame}

\begin{frame}[fragile]
  \frametitle{pthread-test.c}
  \begin{itemize}
  \item Запускаем пять нитей, последняя из которых
    \code{f2()}
\begin{lstlisting}
for (i = 0; i < 5; i++)
{
    pthread_create(&threads[i],
                   NULL, 
                   ((i + 1) % 5) ? f2 : f1, 
                   (void *)i);
}
\end{lstlisting}
\item \code{f1()} и \code{f2()} — рабочие функции
\begin{lstlisting}
void *f1(void *thread_id) { while (1); }
void *f2(void *thread_id) { /* bug here */ }
\end{lstlisting}
  \end{itemize}
\end{frame}

\section{Emacs}
\begin{frame}
  \frametitle{GNU Emacs}

  \begin{itemize}
  \item Расширяемая настраиваемая самодокументируемая операционная
    среда
  \item Компактное переносимое ядро на C
  \item Большая часть функций описана на Emacs Lisp
  \item Широкие возможности по взаимодействию с окружающей средой
    (сеть, IPC)
  \end{itemize}
\end{frame}

\begin{frame}
  \frametitle{Emacs как IDE}

  \begin{itemize}
  \item GDB запускается как подчинённый процесс
  \item Обмен командами идёт с помощью Machine Interface
  \end{itemize}
\end{frame}

\begin{frame}
  \frametitle{CLI vs. MI}
  
\end{frame}

\begin{frame}
 \frametitle{Пример работы с MI}

Foobar
\end{frame}

\section{GSoC}
\begin{frame}
  \frametitle{Google Summer of Code}

  \begin{itemize}
  \item Международная летняя программа для студентов
  \item Под руководством ментора из числа опытных разработчиков
    студент работает над проектом в области свободного \textsc{ПО}
  \item С 2005 года в программе успешно приняли участие 2500 студентов
    из 98 стран
  \end{itemize}
\end{frame}

\begin{frame}
  \frametitle{Этапы GSoC}
  \begin{itemize}
  \item В начале года \neword{организации} подают заявки на участие,
    на основе которых их принимают в программу и распределяют слоты
    для участников
  \item В марте \neword{Студенты} подают заявки с описанием своих
    \neword{проектов}, которые рассматриваются организациями
  \item На основе заявок определяется список студентов, которые будут
    участвовать в программе, назначаются \neword{менторы}
  \item Основная работа начинается в конце мая
  \item В середине июля — \neword{смотр}
  \item Проекты проходят \neword{финальный смотр} в середине августа
  \end{itemize}
\end{frame}

\begin{frame}
  \frametitle{Tips \& Tricks}
  \begin{itemize}
  \item Успешнее оказываются проекты тех студентов, которые общались
    со своими организациями до программы
    
  \item Заявка на участие должна отражать готовность работать над
    проектом

  \item План работ помогает разобраться в задаче
    
  \item Постоянная работа с собственными мыслями и наблюдениями
    
  \item Нужно уделять внимание взаимодействию с сообществом
  \end{itemize}
\end{frame}
\appendix
\begin{frame}
  \frametitle{О чём был доклад}
  \begin{itemize}
  \item Свободный отладчик GDB развивается и отвечает на вызовы
    времени
  \item Emacs — гибкая и многофункциональная рабочая среда
  \item Лето с Google — прекрасная возможность получить опыт и
    принести пользу сообществу и себе
  \end{itemize}
\end{frame}
\end{document}